\textbf{4.8[*]. Show exactly where in our implementation of the store these operations take linear time rather than constant time.}
\begin{itemize}
    \item Procedure like \texttt{setref!} take linear time because it will go throug the list in store.
\end{itemize}

\textbf{4.9[*]. Implement the store in constant time by representing it as a Scheme vector. What is lost by using this representation?}
\begin{itemize}
    \item Solution implemented in ex4-9.rkt
    \item \texttt{newref} still in linear time. Allocating locations in constant time is posible by preallocating more locations in advance, but is more complicated. 
    \item Disadvantage of using Scheme vector to implement the store is that sharing values can be difficult.
\end{itemize}

\textbf{4.10[*]. Implement the \texttt{begin} expression as specified in exercise 4.4.}
\begin{itemize}
    \item Solution implemented in ex4-10.rkt
\end{itemize}

\textbf{4.11[*]. Implement \texttt{list} from exercise 4.5}
\begin{itemize}
    \item Solution implemented in ex4-11.rkt
\end{itemize}



